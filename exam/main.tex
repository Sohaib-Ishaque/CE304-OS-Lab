\documentclass[12pt]{exam}
\usepackage[utf8]{inputenc}

\usepackage[margin=1in]{geometry}
\usepackage{amsmath,amssymb}
\usepackage{multicol}

\newcommand{\class}{Operating System Laboratory}
\newcommand{\term}{Fall 2018}
\newcommand{\examnum}{Final}
\newcommand{\examdate}{18 Dec 2018}
\newcommand{\timelimit}{30 Minutes}

\pagestyle{head}
\firstpageheader{}{}{}
\runningheader{\class}{\examnum\ - Page \thepage\ of \numpages}{\examdate}
\runningheadrule


\begin{document}

\noindent
\begin{tabular*}{\textwidth}{l @{\extracolsep{\fill}} r @{\extracolsep{6pt}} l}
\textbf{\class} & \textbf{Name:} & \makebox[2in]{\hrulefill}\\
\textbf{\term} & \textbf{Student Number:} & \makebox[2in]{\hrulefill}\\
\textbf{\examnum} &&\\
\textbf{\examdate} &&\\
\textbf{Time Limit: \timelimit}
\end{tabular*}\\
\rule[2ex]{\textwidth}{2pt}

This exam contains \numpages\ pages (including this cover page) and \numquestions\ questions.\\
Total of points is \numpoints.

Keep clam and take your exam. Please consider C programming language and Linux operating system.
This exam has only 5 points.

\begin{center}
Grade Table (for teacher use only)\\
\addpoints
\gradetable[v][questions]
\end{center}

\noindent
\rule[2ex]{\textwidth}{2pt}

\begin{questions}

\question[2] What is the difference between interrupt and trap?
\makeemptybox{\fill}
\addpoints

\newpage

\question[1] A hardware device wishes to transfer information to the main memory of the
computer for access by an application. Which of the following mechanisms \textbf{cannot} be used
to this purpose?
\begin{choices}
        \choice direct memory access
        \choice polled mode operation
        \choice programmed I/O
        \choice zombie processes
\end{choices}

\question[1] Which one of the following system calls is related to process creation?
\begin{choices}
        \choice execl
        \choice read
        \choice write
        \choice send
\end{choices}

\question[1] Which one of the following functions can be used as a thread handler?
\begin{choices}
        \choice void *handler(void *)
        \choice int handler(int, int)
        \choice void *handler(char *, int *)
        \choice boolean handler(string, string)
\end{choices}

\question[1] Which choice is not a way for thread synchronization?
\begin{choices}
        \choice mutex
        \choice semaphore
        \choice spinlock
        \choice pipe
\end{choices}

\question[1] Is there any way to have a shared value in multiple process?
\begin{choices}
        \choice Yes
        \choice Yes but it works only after the value is changed
        \choice Yes but it does not reflect the changing
        \choice No
\end{choices}

\question[1] Which relationship is meaningful?
\begin{choices}
        \choice pipe and threads
        \choice fifo and threads
        \choice sockets and processes
        \choice sockets and threads
\end{choices}

\newpage

\question[1] Linux only supports IP in network layer?
\begin{choices}
        \choice true
        \choice false
\end{choices}

\question[1] POSIX is a standard that defines how network protocol works?
\begin{choices}
        \choice true
        \choice false
\end{choices}

\question[0]
What do you think about our class and this exam? Do you consider our class and topics
useful?
\makeemptybox{\fill}

\end{questions}

\end{document}
